\documentclass[a4paper]{article}

\usepackage[top=3cm,bottom=3cm,left=3cm,right=3cm]{geometry}

\title{Abstract OS 2012} 

\author{Tom Peerdeman\\Student nr.: 10266186}

\begin{document}
\maketitle
\vspace{2cm}
\hspace{-2em}
Met de groei van het aantal processoren steeg de vraag naar een nieuw besturingsysteem, hiervoor is Amoeba ontwikkeld. Amoeba is een besturingssysteem voor een gedistribueerde systemen. De vraag is wat is de status van dit besturingssysteem en hoe kan het verbeterd worden. Amoeba gebruikt een microkernel, welke hoewel eerst gedacht te langzaam te zijn toch prima werkt. Verder heeft Amoeba een object geori\"enteerd model welke makkelijk servers liet ontwikkelen.  Een verbeterpunt is het gebruik van de RPC�s. Deze zijn namelijk niet geschikt voor groepscommunicatie. Een ander verbeterpunt is  het feit dat Amoeba geen pre-emption  heeft, hierdoor konden kritieke secties gebroken worden bij een RPC. Vergeleken met andere gedistribueerde systemen bleek Amoeba het goed te doen. Amoeba had een van de snelste reactietijden bij een lege RPC.  Verder bleek de doorvoersnelheid hoger te liggen dan bij de andere systemen wanneer deze gecorrigeerd werden voor user-to-user RPC�s.

\end{document}