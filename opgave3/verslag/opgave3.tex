\documentclass[11pt]{article}
\usepackage{fancyhdr}
\usepackage[usenames, dvipsnames]{xcolor}
\usepackage{graphicx,hyperref}
\hypersetup{
	colorlinks,
	citecolor=black,
	filecolor=black,
	linkcolor=black,
	urlcolor=black
}
\newcommand{\HRule}{\rule{\linewidth}{0.5mm}}
\pagestyle{fancy}
\lfoot{\small \color{gray}Tom Peerdeman - 10266186}
\cfoot{\thepage}
\rfoot{\small \color{gray}Ren\'e Aparicio Sa\'ez - 10214054}
\lhead{\small \color{gray}Opgave 3: Geheugenbeheer}
\begin{document}
	\begin{titlepage}
	\begin{center}
		\textsc{\Large Besturingssystemen}\\[0.5cm]
		\HRule \\[0,4cm]
		\textsc{\huge \bfseries Geheugenbeheer}
		\HRule \\[8cm]
		\begin{minipage}{0.4\textwidth}
			\begin{flushleft}\large
				\emph{Auteurs: Tom Peerdeman \& Ren\'e Aparicio Saez}\\
			\end{flushleft}
		\end{minipage}
		\begin{minipage}{0.4\textwidth}
			\begin{flushright}\large
			\emph{Datum: 25-05-2012\\\'}\\
			\end{flushright}
		\end{minipage}
	\end{center}
	\end{titlepage}

	\tableofcontents
	\newpage

	\section{Inleiding}\label{sec:inleiding}

	\section{Makefile / setup}\label{sec:makefile}
	De makefile maakt twee programma's: worst-fit en first-fit. Om de programma's te maken dient het commando 'make' uitgevoerd te worden. De programma's worden dan gecompileerd. Voor gebruiksgemak is het commando 'make run' toegevoegd. Als dit commando wordt gegeven worden beide programma's direct uitgevoerd. De input die aan de programma's meegegeven wordt staat in de file 'input.txt' in de directory 'input'. De output die het programma geeft wordt weggeschreven naar 'worst-fit.txt' of 'firsti-fit.txt' in de directory out. Om de output snel te verwijderen kan het commando 'make cleanout' worden uitgevoerd. De uitvoer bestanden worden dan verwijderd. Om alle gecompileerd en gelinkt files te verwijderen kan het commando 'make clean' gebruikt worden. Om zowel de outputfiles als de gecompileerde en gelinkte files te verwijderen kan het commando 'make cleanall' gebruikt worden.

	\section{Resultaten}\label{sec:resultaten}


\end{document}
